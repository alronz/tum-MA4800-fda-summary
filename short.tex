\documentclass{article}
\usepackage{amsmath}
\usepackage{amssymb}
\usepackage{amsthm}
\usepackage{enumitem}
\usepackage{graphicx}
\usepackage{amsfonts}
\usepackage{caption}
\usepackage{fancyhdr}
\usepackage{tikz}
\usepackage{xcolor}
\usepackage{hyperref} 

\begin{document}

\newpage
If you want to edit or correct this summary, you can find the source LaTeX code here: \href{https://github.com/alronz/tum-MA4800-fda-summary}{https://github.com/alronz/tum-MA4800-fda-summary}







\newpage


\section{General}

\begin{itemize}

\item 

$ \log_b (a^c) = c \log_b (a) $

\item \textbf{Unitary Matrix:} \( A^T A = AA^T = I, A^T = A^{-1}, A^{+} = A^{-1}, A^{+} = A^T \).
\item Homogeneity: $\|\lambda v\| = |\lambda|\|v\|$
\item Triangle Inequality: $\|v + w\| \leq \|v\| + \|w\|$

\item Inverse Triangle Inequality: $\| \|v\| - \|w\| \leq \|v - w\|$

\item \textbf{Linearity in First Argument:} $\langle u + \lambda v, w \rangle = \langle u, w \rangle + \lambda \langle v, w \rangle$
\item \textbf{Linearity in Second Argument:} $\langle w, u + \lambda v \rangle = \langle w, u \rangle + \lambda \langle w, v \rangle$

\item $\|v\| = \sqrt{\langle v, v \rangle}$

\item $|\langle v, w \rangle| \leq \|v\| \|w\|$

 \item 
    $ \|v\|_2 = \sqrt{\sum_{i \in I} |v_i|^2} $

\item Orthonormal Set: $\|v_i\| = 1 \text{ and } \langle v_i, v_j \rangle = 0 \text{ for } i \neq j$

\item \textbf{Cauchy Sequence:} is a sequence $\{v_n\}$ where $\|v_n - v_m\|$ becomes arbitrarily small for sufficiently large $n$ and $m$.

\item $C$ is convex if $tv + (1 - t)w \in C$ for all $v, w \in C$ and $t \in [0, 1]$.

\item $P_C(v) = \arg \min_{w \in C} \|v - w\|$

\item $\langle z - P_C(v), v - P_C(v) \rangle \leq 0$ for all $z \in C$.

 \item If \( W \) has an orthonormal basis \( \{ w_\nu \}_{\nu \in F} \):
    \[
    P_W(v) = \sum_{\nu \in F} \langle v, w_\nu \rangle w_\nu,
    \]

 \item For \( v \in V \):
    \[
    \| P_W(v) \|^2 = \sum_{\nu \in F} |\langle v, w_\nu \rangle|^2.
    \]

 \item $P_V(v) = v$ when $W = V$.

 \item $v = \sum_{\nu \in F} \langle v, w_\nu \rangle w_\nu$.

 \item $\| v \|^2 = \sum_{\nu \in F} |\langle v, w_\nu \rangle|^2$.

 \item Linearity: $\text{tr}(\alpha A + \beta B) = \alpha \text{tr}(A) + \beta \text{tr}(B)$
    \item Cyclic Property: $\text{tr}(AB) = \text{tr}(BA)$
    \item Sum of Eigenvalues: $\text{tr}(A) = \sum_{i=1}^{n} \lambda_i$
 
\item \textbf{Frobenius Norm:}
    \[
    \| A \|_F = \sqrt{\sum_{i \in I} \sum_{j \in J} |A_{ij}|^2}
    \]

 \item 
    $
    \| A \|_F^2 = \text{tr}(AA^H) = \text{tr}(A^H A)
   $

\item 
$
\|A\| = \|A\|_{X \rightarrow Y} = \sup_{z \neq 0} \frac{\|Az\|_Y}{\|z\|_X}
$

\item 
$
\|AB\|_F \leq \|A\|_F \|B\|_F
$

\item $
\|AB\| \leq \|A\| \|B\|
$


\item PARALLELOGRAM LAW

\[
\|x + y\|^2_2 = 2 \|x\|^2_2 + 2 \|y\|^2_2 - \|x - y\|^2_2
\]

\[
\|x - y\|^2_2 = 2 \|x\|^2_2 + 2 \|y\|^2_2 - \|x + y\|^2_2
\]

\[
(x - y)^2 = 2x^2 + 2y^2 - (x + y)^2
\]

\[ \|x + y\|_2^2 = \|x\|_2^2 + \|y\|_2^2 + 2 \langle x, y \rangle \]

\[ \|x - y\|_2^2 = \|x\|_2^2 + \|y\|_2^2 - 2 \langle x, y \rangle \]


\item Taylor Expansions
\begin{itemize}

\item First-Order:
\[
f(x_0 + t) = f(x_0) + f'(x_0) \cdot t + o(t), \quad \text{as } t \to 0
\]
\[
\lim_{t \to 0} \frac{o(t)}{t} = 0
\]

\item Second-Order Taylor
   \[
        f(x_0 + t) = f(x_0) + f'(x_0) \cdot t + \frac{1}{2} f''(x_0) \cdot t^2 + o(t^2), \quad \text{as } t \to 0
        \]
    \[
        \lim_{t \to 0} \frac{o(t^2)}{t^2} = 0
        \]
\end{itemize}

\item Taylor Expansion with Mean Value:
\begin{itemize}
    \item \textbf{First-Order:}
    \[
    f(x_0 + t) = f(x_0) + f'(z) \cdot t, \quad z \in (x_0, x_0 + t)
    \]

    \item \textbf{Second-Order }
    \[
    f(x_0 + t) = f(x_0) + f'(x_0) \cdot t + \frac{1}{2} f''(z) \cdot t^2, \quad z \in (x_0, x_0 + t)
    \]

    \item Multidimentional 
\[
f(x_0 + ty) = f(x_0) + t \nabla f(x_0) \cdot y + \frac{1}{2} t^2 y^T \nabla^2 f(x_0) y + o(t^2), \quad \text{as } t \to 0
\]



\end{itemize}

\item  $
    P_V = V V^T \in \mathbb{R}^{d \times d}
    $
\item   $
    P_V \cdot x = P_V(x)
    $
\item $
(f * g)(t) = \int_{-\infty}^{\infty} f(u)g(t - u) \, du
$

\item $
\phi_{X+Y}(t) = \left( \phi_X * \phi_Y \right)(t) = \int_{-\infty}^{\infty} \phi_X(u)\phi_Y(t - u) \, du
$



\item Quadratic Function: $
f(x) = x^T A x + \langle b, x \rangle + c
$ :

\begin{itemize}
    \item $
\nabla f(x) = Ax + A^T x + b
$
(For symmetric \(A\): \(\nabla f(x) = 2Ax + b\))
\item $
\nabla^2 f(x) = A + A^T
$
(For symmetric \(A\): \(\nabla^2 f(x) = 2A\))

\end{itemize}

\item $
\nabla \|x\|_2 = \frac{x}{\|x\|_2} \text{ for } x \neq 0.
$

\item $
\nabla \|x\|^2_2 = 2x.
$

\item $
B^m = \sum_{k=1}^{r} \sigma_k^{2m} u_k u_k^T
$

\end{itemize}



\section{SVD}

\begin{itemize}

\item SVD

\[
A = \sum_{i=1}^r \sigma_i u_i v_i^T
\]
\[
A = U \Sigma V^T
\]
\[
U^T U = I
\]
\[
U U^T = I
\]
\[
V^T V = I
\]
\[
V V^T = I
\]
\[
\Sigma \Sigma^{-1} = I
\]
\[
\Sigma^{-1} \Sigma = I
\]

\item 
$
\sigma_1(A) = \|Av_1\|_2
$


\item 
$
v_k = \arg \max_{\|x\|_2=1,\; \langle v_1,v_2,\ldots,v_{k-1} \rangle=0} \|Av\|_2
$

\item    $
    \| A^{(i)} \|_2^2 = \sum_{k=1}^{r} | \langle A^{(i)}, v_k \rangle |^2
    $

\item 
$
    \sum_{i \in I} \| A^{(i)} \|_2^2 = \sum_{k=1}^{r} \| A_v k \|_2^2 = \sum_{k=1}^{r} \sigma_k(A)^2
$

\item
   $
    \| A \|_F^2 = \sum_{i \in I} \sum_{j \in J} | A_{ij} |^2 = \sum_{k=1}^{r} \sigma_k(A)^2
  $

\item 
    $
    \| A \|_F = \sqrt{ \sum_{k=1}^{r} \sigma_k(A)^2 }
   $

\item 
$
\|A\| = \sigma_1(A) = \max_{\|v\|_2=1} \|Av\|_2
$

\item Frobenius Norm ($p = 2$):
        \[
        \|A\|_F = \left( \sum_{k=1}^r \sigma_k(A)^2 \right)^{1/2}
        \]

        \item Spectral Norm ($p = \infty$):
        \[
        \|A\|_\infty = \max_{k=1,\ldots,r} \sigma_k(A) = \sigma_1(A)
        \]

        \item Nuclear Norm ($p = 1$):
        \[
        \|A\|_* = \|A\|_1 = \sum_{k=1}^r \sigma_k(A)
        \]

\item A matrix $A$ is positive semi-definite if:
        \[
        \langle x, Ax \rangle \geq 0 \quad \forall x \in \mathbb{R}^m.
        \]

 \item A matrix $A$ is positive definite if:
        \[
        \langle x, Ax \rangle > 0 \quad \forall x \in \mathbb{R}^m, \, x \neq 0.
        \]



\item 
   $
    A^+ = V \Sigma^{-1} U^H
   $


\item 
$
A^+ = \sum_{k=1}^{r} \sigma_k^{-1} v_k u_k^H
$


\item if $(m > n)$ (long matrix)

    $
    A^+ = (A^H A)^{-1} A^H
    $ and $x = A^+y = (A^H A)^{-1} A^Hy$

\item if $(m < n)$ (fat matrix)

$
 A^+ = A^H(AA^H)^{-1}
$ and $x = A^+y = A^H(AA^H)^{-1}y$

\item Weyl's Bounds

\[
\lambda_k(A) + \lambda_n(E) \leq \lambda_k(A + E) \leq \lambda_k(A) + \lambda_1(E)
\]

\[
\lvert \sigma_k(A + E) - \sigma_k(A) \rvert \leq \lVert E \rVert
\]


\item Mirsky's Bounds
\[
\sum_{k=1}^{n} \lvert \sigma_k(A + E) - \sigma_k(A) \rvert^2 \leq \lVert E \rVert_F^2
\]

\item   $
    \cos \theta(V,W) = \frac{V^T W}{\lVert V \rVert_2 \lVert W \rVert_2}
    $

\end{itemize}


\section{Probability}

\begin{itemize}

\item Random variable X has PDF 

    \[
    P(a < X \leq b) = \int_a^b \varphi(t) \, dt \quad \text{for all } a < b \in \mathbb{R}
    \]

\item \textbf{Relationship with Distribution Function:}
\[
\varphi(t) = \frac{d}{dt} F(t)
\]
where the cumulative distribution function (CDF) \( F(x) \) is defined as:
\[
F(x) = P(X \leq x) = \int_{-\infty}^{x} \frac{d}{dt} F(t) \, dt
\]


\item Expectation (Mean)

 \[
    E[X] = \int_{\Omega} X(\omega) \, dP(\omega)
    \]

\[
        E[g(X)] = \int_{-\infty}^{\infty} g(t) \varphi(t) \, dt
        \]

    \[
    E[X] = \int_{-\infty}^{\infty} t \varphi(t) \, dt
    \]

\item Moments
    \[
    E[X^p] \quad \text{for } p > 0
    \]
\item Absolute Moments
    \[
    E[|X|^p] \quad \text{for } p > 0
    \]

\item variance
    \[
    Var(X) = E[(X - E[X])^2] = E[X^2] - (E[X])^2
    \]

\item LP Norm:
\[
\|X\|_p = (E[|X|^p])^{1/p} \quad \text{for } 1 \leq p < \infty
\]

\item Inequality:
\[
\|X + Y\|_p \leq \|X\|_p + \|Y\|_p
\]

\item Hölder's Inequality

    \[
    |E[XY]| \leq \|X\|_p \|Y\|_q
    \]

\item Convergence of Random Variables

\[
\lim_{n \to \infty} X_n(\omega) = X(\omega) \quad \text{for all } \omega \in \Omega
\]

\item Lebesgue's Dominated Convergence Theorem (LDCT)

\[
\lim_{n \to \infty} E[X_n] = E\left[\lim_{n \to \infty} X_n\right] = E[X]
\]

\item Fubini's Theorem
\[
\int_A \left( \int_B f(x, y) d\mu(y) \right) d\nu(x) = \int_B \left( \int_A f(x, y) d\nu(x) \right) d\mu(y)
\]

\item Absolute Moments

    \[
    E[|X|^p] = p \int_0^{\infty} P(|X| \ge t) t^{p-1} dt
    \]

 \item $\left( \mathbb{E}|X+Y| \right)^{p} \leq \left( \mathbb{E}|X|^{p} \right) + \left( \mathbb{E}|Y|^{p} \right)$

\item Cavalieri's Formula for Expectation

    \[
    E[X] = \int_0^{\infty} P(X \ge t) dt - \int_0^{\infty} P(X \le -t) dt
    \]

\item Markov's Inequality

    \[
    P(|X| \ge t) \le \frac{E[|X|]}{t}
    \]

\item Generalized Markov Inequality

    \[
    P(|X| \ge t) = P(|X|^p \ge t^p) \le \frac{E[|X|^p]}{t^p} \quad \text{for all } t > 0
    \]

\item Chebyshev Inequality

    \[
    P(|X - E[X]| \ge t) \le \frac{Var(X)}{t^2} \quad \text{for all } t > 0
    \]

\item Laplace Transform (Moment Generating Function)

    \[
    M_X(\theta) = E[e^{\theta X}]
    \]

\item Normal Distribution (Gaussian Distribution)

\begin{itemize}
    \item 
  \[
    \psi(t) = \frac{1}{\sqrt{2\pi\sigma^2}} \exp\left(-\frac{(t - \mu)^2}{2\sigma^2}\right)
    \]

     \item MGF: $M_X(\theta) = \exp\left(\mu \theta + \frac{1}{2} \sigma^2 \theta^2\right)$
\item \(\mu\): Mean
\item \(\sigma^2\): Variance
 \item Mean: \(E[X] = \mu\)
        \item Variance: \(\text{Var}(X) = \sigma^2\)

        \[
    X \sim N(\mu, \sigma^2)
    \]

\item \textbf{Expectation:}
\[
\mathbb{E}[X] = \frac{1}{N} \sum_{i=1}^{N} X_i
\]

\item \textbf{Variance:}
\[
\text{Var}(X) = \mathbb{E}[X^2] - (\mathbb{E}[X])^2
\]
Alternatively,
\[
\text{Var}(X) = \frac{1}{N} \sum_{i=1}^{N} X_i^2 - \left( \frac{1}{N} \sum_{i=1}^{N} X_i \right)^2
\]

\item \textbf{Standard Deviation:}
\[
\sigma_X = \sqrt{\text{Var}(X)}
\]
\end{itemize}

  \item \textbf{for normal distribution PDF:}

\begin{itemize}
\item  $
    \phi(t) = \frac{1}{\sqrt{2\pi}} \exp\left(-\frac{t^2}{2}\right)
    $
      \item MGF: $M_X(\theta) = \exp\left(\frac{1}{2} \theta^2\right)$
\end{itemize}
   

\item 
    $
    P(X_1 \leq t_1, \ldots, X_n \leq t_n) = \prod_{\ell=1}^{n} P(X_\ell \leq t_\ell)
    $

\item 

    $
    E\left[ \prod_{\ell=1}^{n} X_\ell \right] = \prod_{\ell=1}^{n} E[X_\ell]
    $

\item join PDF

$
    \varphi(t_1, \ldots, t_n) = \varphi_1(t_1) \cdot \varphi_2(t_2) \cdots \varphi_n(t_n)
  $

\item PDF of X+Y

$
    \varphi_{X+Y}(t) = (\varphi_X * \varphi_Y)(t) = \int_{-\infty}^{\infty} \varphi_X(u) \varphi_Y(t - u) \, du
$

\item Fubini's Theorem for Expectations

    \[
    E[|f_1(X)|] = E[|f_2(Y)|] = E[|f(X,Y)|]
    \]

\item Multivariate Normal Distribution (Gaussian Vector)

    \[
    X = A g + \mu
    \]

    \[
    \Sigma = A A^T = E[(X - \mu)(X - \mu)^T]
    \]

        \[
    \psi(x) = \frac{1}{(2\pi)^{n/2} \sqrt{\det(\Sigma)}} \exp\left(-\frac{1}{2} (x - \mu)^T \Sigma^{-1} (x - \mu)\right)
    \]


\item Jensen's Inequality

\begin{itemize}
    \item \textbf{For Convex Functions:} 
    \[
    f(E[X]) \leq E[f(X)]
    \]
    \item \textbf{Convex Function:} \( f(\lambda x + (1 - \lambda)y) \leq \lambda f(x) + (1 - \lambda)f(y) \) for all \( x, y \in \mathbb{R}^n \) and \( \lambda \in [0, 1]. \)

    \item \textbf{For Concave Functions:} 
    \[
    E[f(X)] \leq f(E[X])
    \]
    \item \textbf{Concave Function:} \( f(\lambda x + (1 - \lambda)y) \geq \lambda f(x) + (1 - \lambda)f(y) \) for all \( x, y \in \mathbb{R}^n \) and \( \lambda \in [0, 1]. \)
\end{itemize}


\item Moment Generating Function (MGF)

\[
M_X(\theta) = E[\exp(\theta X)]
\]

\item Cumulant Generating Function (CGF)

\[
K_X(\theta) = \ln(E[\exp(\theta X)])
\]

\item Cramer's Theorm

\[
P\left(\sum_{l=1}^{M} X_\ell \geq t\right) \leq \exp\left(\inf_{\theta > 0}\left\{-\theta t + \sum_{l=1}^{M} C_X(\theta)\right\}\right)
\]

\item Hoeffding's Inequiality

\begin{itemize}

\item One-Sided Bound:
    \[
    \mathbb{P}\left(\sum_{\ell=1}^{M} X_\ell \geq t\right) \leq \exp\left(-\frac{t^2}{2\sum_{\ell=1}^{M} B_\ell^2}\right)
    \]

    \item Two-Sided Bound:
    \[
    \mathbb{P}\left(\left|\sum_{\ell=1}^{M} X_\ell\right| \geq t\right) \leq 2 \exp\left(-\frac{t^2}{2\sum_{\ell=1}^{M} B_\ell^2}\right)
    \]

    \item For a single symmetric Bernoulli random variable, Hoeffding's inequality states:

\[
\mathbb{P} \left( |X| > t \right) \leq 2 \exp \left( -\frac{2t^2}{(b-a)^2} \right)
\]

\end{itemize}

\item Bernstein's Inequality

\begin{itemize}[leftmargin=*]
    \item For all $t > 0$:
    \[
    P\left( \left| \sum_{\ell=1}^{M} X_\ell \right| \geq t \right) \leq 2 \exp\left( -\frac{t^2}{2(\sigma^2 + Rt)} \right)
    \]
    \item Where $\sigma^2 = \sum_{t=1}^{M} \sigma_t^2$.
\item when random variables are Subexponential

\[
P\left(\left| \|X\|_2^2 - n \right| > t \right) \leq 2 \exp \left( - \frac{(kt)^2}{2(2\beta n + kt)} \right)
\]


\end{itemize}

\item Johnson-Lindenstrauss Lemma 



\[
k \geq \beta \epsilon^{-2} \log(2n)
\]

       \[
        (1 - \epsilon) \|v - w\|^2_2 \leq \|f(v) - f(w)\|^2_2 \leq (1 + \epsilon) \|v - w\|^2_2 \quad \text{for all } v, w \in P
         \]



\[ |\langle f(\mathbf{v}), f(\mathbf{w}) \rangle| \leq |\langle \mathbf{v}, \mathbf{w} \rangle (1 + \epsilon)| \]


\end{itemize}

\section{Optimization}

\begin{itemize}

\item To find global minimized using gradient decent, following conditions needed:

\begin{itemize}
   \item \textbf{Convexity}: \( f \) is convex if \(\forall \mathbf{x}, \mathbf{y} \in \mathbb{R}^n \) and \( \lambda \in [0,1] \):
   \[
   f(\lambda \mathbf{x} + (1 - \lambda) \mathbf{y}) \leq \lambda f(\mathbf{x}) + (1 - \lambda) f(\mathbf{y}).
   \]

\item \textbf{Lipschitz Continuity of Gradient}: There exists \( L > 0 \) such that \(\forall \mathbf{x}, \mathbf{y} \in \mathbb{R}^n \):
   \[
   \| \nabla f(\mathbf{x}) - \nabla f(\mathbf{y}) \| \leq L \| \mathbf{x} - \mathbf{y} \|.
   \]

\item \textbf{Step Size}: The step size \( \alpha_k \) satisfies:
   \[
   0 < \alpha < \frac{2}{L} \quad 
   \]
\end{itemize}



 \item \textbf{Convex Set:} A set \( K \subseteq \mathbb{R}^N \) is convex if for all \( x, z \in K \) and \( t \in [0, 1]: \quad t \cdot x + (1 - t) \cdot z \in K \)

 \item \textbf{Convex Hull:} For \( T \subseteq \mathbb{R}^N \), the convex hull \( \text{conv}(T) \) is the smallest convex set containing \( T \).


\item \textbf{Cones:} A set $K \subseteq \mathbb{R}^N$ is a cone if for all $x \in K$ and all $t \geq 0$, $t \cdot x \in K$.


\item \textbf{Convex Cone:} A set $K$ is a convex cone if it is a cone and convex. For all $x,z \in K$ and $s, t \geq 0$, $s \cdot x + t \cdot z \in K$.

\item \textbf{Dual Cones:} For a cone $K \subseteq \mathbb{R}^N$, the dual cone $K^*$ is:
    \[
    K^* = \{z \in \mathbb{R}^N : \langle x, z \rangle \geq 0 \text{ for all } x \in K\}
    \]

\item Bidual Cone: If $K$ is a closed cone, then $(K^*)^* = K$.

\item Polar Cones:
For a cone \( K \subseteq \mathbb{R}^N \), the polar cone \( K^\circ \) is:
\[
K^\circ = \{ z \in \mathbb{R}^N : \langle x, z \rangle \leq 0 \text{ for all } x \in K \}
\]

\item \( K^\circ = -K^* \), where \( K^* \) is the dual cone.


\item Geometrical Hahn-Banach Theorem

For convex sets \( K_1, K_2 \subseteq \mathbb{R}^N \) with empty interior intersection, there exists a vector \( w \in \mathbb{R}^N \) and a scalar \( \lambda \) such that:
\[
K_1 \subseteq \{ x \in \mathbb{R}^N : \langle x, w \rangle \leq \lambda \}
\]
\[
K_2 \subseteq \{ x \in \mathbb{R}^N : \langle x, w \rangle \geq \lambda \}
\]


\item Defined as $\text{dom}(F) = \{x \in \mathbb{R}^N : F(x) \neq \infty\}$.

\item A function is proper if $\text{dom}(F) \neq \emptyset$.

\item Convex Functions

\( F : \mathbb{R}^N \to (-\infty, \infty) \) is convex if for all \( x, z \in \mathbb{R}^N \) and \( t \in [0, 1] \):
\[
F(tx + (1 - t)z) \leq tF(x) + (1 - t)F(z)
\]

\item Strictly Convex Functions
\( F \) is strictly convex if for all \( x \neq z \) and \( t \in (0,1) \):
\[
F(tx + (1 - t)z) < tF(x) + (1 - t)F(z)
\]

\item Strongly Convex Functions
\( F \) is strongly convex with parameter \( \gamma > 0 \) if for all \( x, z \in \mathbb{R}^N \) and \( t \in [0, 1] \):
\[
F(tx + (1 - t)z) \leq tF(x) + (1 - t)F(z) - \frac{\gamma}{2}(1 - t)\|x - z\|_2^2
\]

\item 
The epigraph of $F$ is $\operatorname{epi}(F) = \{(x, r) : r \geq F(x)\}$.

\begin{itemize}
        \item $F$ is convex if and only if $\operatorname{epi}(F)$ is a convex set.
        \item For $(x_1, r_1), (x_2, r_2) \in \operatorname{epi}(F)$ and $t \in [0, 1]$:
        \[
        (tx_1 + (1 - t)x_2, tr_1 + (1 - t)r_2) \in \operatorname{epi}(F)
        \]
    \end{itemize}

\item A differentiable function $F : \mathbb{R}^N \rightarrow \mathbb{R}$ is convex if and only if:
    \[
    F(x) \geq F(y) + \nabla F(y)^T (x - y)
    \]

 \item $F$ is strongly convex with parameter $\gamma > 0$ if and only if:
    \[
    F(x) \geq F(y) + \nabla F(y)^T (x - y) + \frac{\gamma}{2} \| x - y \|^2
    \]

\item A twice differentiable function $F$ is convex if and only if its Hessian $\nabla^2 F(x)$ is positive semi-definite:
    \[
    \nabla^2 F(x) \succeq 0
    \]

\item to check convexity:

\begin{itemize}
    \item Convexity can be checked using the gradient. If the tangent line (or hyperplane) at any point $y$ lies below the function, $F$ is convex. Condition from above.
    \item \textbf{Strong Convexity:} Strong convexity includes an additional quadratic term that provides a lower bound on the curvature of $F$.
    \item \textbf{Hessian Condition:} For twice differentiable functions, convexity can be checked by ensuring the Hessian matrix is positive semi-definite at every point.
    
\end{itemize}

    \item Let $F, G$ be convex functions on $\mathbb{R}^N$. Then, for $\alpha, \beta \geq 0$ the function $\alpha F + \beta G$ is convex.
    \item Let $F : \mathbb{R} \rightarrow \mathbb{R}$ be convex and nondecreasing, and $G : \mathbb{R}^N \rightarrow \mathbb{R}$ be convex. Then $H(x) = F(G(x))$ is convex.

\item Lower Semicontinuity:
 \[
    \liminf_{j \to \infty} F(x_j) \geq F(x)
    \]

 \item \textbf{Global Minimum:} A point \( x \in \mathbb{R}^N \) is a global minimum of \( F \) if:
    \[
    F(x) \leq F(y) \quad \text{for all } y \in \mathbb{R}^N
    \]
    \item \textbf{Local Minimum:} A point \( x \in \mathbb{R}^N \) is a local minimum of \( F \) if there exists \( \epsilon > 0 \) such that:
    \[
    F(x) \leq F(y) \quad \text{for all } y \text{ satisfying } \| x - y \|_2 \leq \epsilon
    \]

 \item For a convex function \( F \), any local minimum is also a global minimum.

  \item The set of minima of a convex function \( F \) is convex.

 \item If \( F \) is strictly convex, the minimum is unique.

 \item For extreme point \( x, \ x = ty + (1-t)z \) with \( t \in (0,1) \) implies \( x = y = z \).


 \item Convex Conjugate (\( F^* \) is always a convex function, even if \( F \) is not.)
  \[
    F^*(y) := \sup_{x \in \mathbb{R}^N} \{ \langle x, y \rangle - F(x) \}
    \]

\item Fenchel-Young Inequality: For all \( x, y \in \mathbb{R}^N \):
    \[
    \langle x, y \rangle \leq F(x) + F^*(y)
    \]

\item Biconjugate:
\begin{itemize}
        \item \( F^{**} \) is the largest lower semicontinuous convex function satisfying \( F^{**}(x) \leq F(x) \).
        \item If \( F \) is convex and lower semicontinuous, then \( F = F^{**} \).

         \item Scaling Argument:  For \( \tau \neq 0 \):
        \[
        (F_\tau)^*(y) = F^*\left(\frac{y}{\tau}\right)
        \]

        \item Scaling Function: For \( \tau > 0 \):
        \[
        (\tau F)^*(y) = \tau F^*\left(\frac{y}{\tau}\right)
        \]

        \item Translation: For \( z \in \mathbb{R}^N \):
        \[
        (F_z)^*(y) = \langle z, y \rangle + F^*(y)
        \]
    \end{itemize}



 \item \textbf{Subdifferential and Subgradients:} For a convex function \( F : \mathbb{R}^N \to \mathbb{R} \) at \( x \in \mathbb{R}^N \),
    \[
    \partial F(x) = \{ v \in \mathbb{R}^N : F(z) - F(x) \geq \langle v, z - x \rangle \text{ for all } z \in \mathbb{R}^N \}
    \]

    \( v \) is a subgradient at \( x \) if it satisfies:
    \[
    F(z) - F(x) \geq \langle v, z - x \rangle
    \]

     The subdifferential \( \partial F(x) \) is always non-empty for a convex function.

      If \( F \) is differentiable at \( x \),
    \[
    \partial F(x) = \{ \nabla F(x) \}
    \]

\item  A vector $x$ is a minimum of a convex function $F$ if and only if:
    \[
    0 \in \partial F(x)
    \]

 \item For a convex lower semicontinuous function $F$:
        \[
        x \in \partial F^*(y) \leftarrow \rightarrow y \in \partial F(x)
        \]

\item Proximal Mapping:

\[
\text{prox}_F(z) := \arg \min_{x \in \mathbb{R}^N} \left\{ F(x) + \frac{1}{2} \| x - z \|_2^2 \right\}
\]

The function $x \mapsto F(x) + \frac{1}{2} \| x - z \|_2^2$ is strictly convex, ensuring a unique minimizer.

For a convex function $F : \mathbb{R}^N \rightarrow (-\infty,\infty]$,
\[
x = P_F(z) \text{ if and only if } z \in x + \partial F(x)
\]
\[
P_F = (I + \partial F)^{-1}
\]

\item Moreau's Identity:
For a lower semicontinuous convex function $F : \mathbb{R}^N \rightarrow (-\infty,\infty]$ and all $z \in \mathbb{R}^N$,
\[
P_F(z) + P_{F^*}(z) = z
\]

\item Nonexpansiveness of Proximal Mappings:
\[
\|P_F(z) - P_F(z')\|_2 \leq \|z - z'\|_2 \quad \text{ for all } z, z' \in \mathbb{R}^N.
\]

\item Lagrange Function:
\[
L(x, \xi, \nu) = F_0(x) + \xi^T(Ax - y) + \sum_{l=1}^M \nu_l (F_l(x) - b_l).
\]
\begin{itemize}
    \item \(x \in \mathbb{R}^N\): Decision variables.
    \item \(\xi \in \mathbb{R}^m\): Lagrange multipliers for equality constraints. \textcolor{red}{ It can take any real values, positive or negative}
    \item \(\nu \in \mathbb{R}^M\): Lagrange multipliers for inequality constraints (\(\nu_l \geq 0\)).
\end{itemize}

\item Lagrange Dual Function:
\[
H(\xi, \nu) = \inf_{x \in \mathbb{R}^N} L(x, \xi, \nu),
\]

\begin{itemize}
     \item \textbf{Concavity:} The dual function $H(\xi, \nu)$ is concave.
    \item \textbf{Dual Feasible:} $(\xi, \nu)$ is dual feasible if $\xi \in \mathbb{R}^m$ and $\nu \geq 0$.
    \item \textbf{Dual Optimal:} $(\xi^*, \nu^*)$ maximizes $H(\xi, \nu)$.
    \item \textbf{Primal-Dual Optimal:} $(x^*, \xi^*, \nu^*)$ where $x^*$ is optimal for the primal problem and $(\xi^*, \nu^*)$ are optimal for the dual problem.
    \item Weak Duality: \( H(\xi^{*}, \nu^{*}) \leq F_{0}(x^{*}) \)
    \item Strong Duality: \( H(\xi^{*}, \nu^{*}) = F_{0}(x^{*}) \)
\end{itemize}

\item Slater's Constraint Qualification Theorem:
\begin{itemize}
    \item \textbf{Assumption:} \( F_{0}, F_{1}, \ldots, F_{M} \) are convex functions with \( \text{dom}(F_{0}) = \mathbb{R}^{N} \).
    \item \textbf{Condition:} There exists \( x \in \mathbb{R}^{N} \) such that:
    \[
    Ax = y \quad F_{i}(x) < b_{i}, \quad \forall i \in \{1, \ldots, M\}
    \]
    \item \textbf{Conclusion:} If the above conditions hold, then strong duality holds for the optimization problem.
\end{itemize}

\item Saddle-Point Interpretation:
\begin{itemize}
     \item \textbf{Optimal Value:}
    \[
    F_{0}(x^{*}) = \inf_{x \in \mathbb{R}^{N}} \sup_{\xi \in \mathbb{R}^{m}} L(x, \xi).
    \]
 \item \textbf{Supremum:}
    \[
    \sup_{\xi \in \mathbb{R}^m} \inf_{x \in \mathbb{R}^N} L(x, \xi) \leq \inf_{x \in \mathbb{R}^N} \sup_{\xi \in \mathbb{R}^m} L(x, \xi)
    \]
     \item \textbf{Strong Duality:}
    \[
    \sup_{\xi \in \mathbb{R}^m} \inf_{x \in \mathbb{R}^N} L(x, \xi) = \inf_{x \in \mathbb{R}^N} \sup_{\xi \in \mathbb{R}^m} L(x, \xi)
    \]
    \item For a primal-dual optimal pair \((x^*, \xi^*)\):
\[
L(x^*, \xi) \leq L(x^*, \xi^*) \leq L(x, \xi^*)
\]
    
\end{itemize}

\item Lipschitz Continuity:
\begin{itemize}
    \item Gradient $\nabla f(x)$ is Lipschitz continuous with constant $L$:
    \[
    \|\nabla f(x) - \nabla f(y)\|_2 \leq L\|x - y\|_2
    \]
    \item If $f$ is twice differentiable:
    \[
    \lambda_{\max}(\nabla^2 f(x)) \leq L
    \]
        \item A convex function $f$ satisfies:
    \[
    f(y) \geq f(x) + \nabla f(x)^{T}(y - x)
    \] For all $x, y \in \mathbb{R}^{d}$.
        \item If $\nabla f(x)$ is Lipschitz continuous with constant $L$, then:
    \[
    f(y) \leq f(x) + \nabla f(x)^{T}(y - x) + \frac{L}{2}\|y - x\|^{2}_{2}
    \]
    \item Convergence of Gradient Descent:
    \[
\alpha = \alpha^{(n)} \leq \frac{1}{L}
\]
\end{itemize}

\item \textbf{Update Rule:}
    \begin{equation*}
    x^{(n+1)} = x^{(n)} - \alpha^{(n)} \nabla f(x^{(n)})
    \end{equation*}

\item Strong Convexity:
\begin{itemize}
    \item \textbf{Gradient Condition:}
    \[
    f(y) \geq f(x) + \langle \nabla f(x), y - x \rangle + \frac{\gamma}{2} \|y - x\|^2_2
    \]

    \item \textbf{Gradient Difference Condition:}
    \[
    \langle \nabla f(x) - \nabla f(y), x - y \rangle \geq \gamma \|x - y\|^2_2
    \]

    \item \textbf{Hessian Condition:}
    \[
    w^T \nabla^2 f(x) w \geq \gamma \|w\|^2_2 \quad \forall x, w \in \mathbb{R}^n
    \]
    \begin{itemize}
        \item Alternatively:
        \[
        \nabla^2 f(x) \succeq \gamma I
        \]
    \end{itemize}

    \item \textbf{Jensen's Inequality for Strongly Convex Functions:}
    \[
    f(\lambda x + (1 - \lambda) y) \leq \lambda f(x) + (1 - \lambda) f(y) - \frac{\gamma}{2} \lambda(1 - \lambda) \|x - y\|^2_2
    \]
\end{itemize}

\item Gradient Inequality:   $
    f(x + v) \geq f(x) + \nabla f(x) \cdot v \quad \forall x, v
    $

 \item Use the spectral norm (largest singular value) to estimate the Lipschitz constant.
    \[
    \|\nabla^2 f(x)\| = \text{Largest Singular Value}
    \]

\end{itemize}




\end{document}
